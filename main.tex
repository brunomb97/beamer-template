\documentclass[xcolor=dvipsname]{beamer}

\usetheme{AnnArbor}

\newcommand{\Titulo}{Sweeping process with regular and nonregular sets}
\newcommand{\Autor}{Lionel Thibault}
\newcommand{\Nombre}{Bruno Martinez Barrera}
\newcommand{\Ramo}{Inclusiones Diferenciales}
\newcommand{\Sigla}{MAT472}
\newcommand{\Rol}{2016 10 007 - 5}
\newcommand{\Correo}{bruno.martinez@sansano.usm.cl}
\newcommand{\ProfesorI}{Christopher Hermosilla}

\title{\Titulo}
\subtitle{\Autor}
\author{\Nombre}
\institute{
  Departamento de Matemática\\
  Universidad Técnica Federico Santa María}
\date{\today}

\subject{\Sigla}

\definecolor{Black}{rgb}{0,0,0}
\definecolor{Green}{rgb}{0,0.4392,0.1019}
\definecolor{Grey}{rgb}{0.851,0.851,0.851}

\setbeamercolor{title}{bg=Grey,fg=black}
\setbeamercolor{frametitle}{bg=Grey,fg=black}
\setbeamercolor{palette primary}{bg=Black,fg=black}
\setbeamercolor{palette secondary}{bg=Green,fg=black}
\setbeamercolor{palette tertiary}{bg=Green,fg=black}
\setbeamercolor{palette quaternary}{bg=Green,fg=black}
\setbeamercolor{structure}{bg=Green,fg=black}
\setbeamercolor{section in toc}{bg=black}

\pgfdeclareimage[height=0.75cm]{university-logo}{utfsm.pdf}
\logo{\pgfuseimage{university-logo}}

\usepackage[brazil,spanish]{babel}

\usepackage{changepage}
\usepackage{graphicx}
\usepackage{graphics}
\usepackage{subfiles}
\usepackage{array}
\usepackage{enumitem}
\usepackage{multirow}
\usepackage{float}
\usepackage{calc}

\usepackage{bbm}
\usepackage[notquote]{hanging}
\usepackage{amsmath}
\usepackage{mathrsfs}
\usepackage{amsfonts}
\usepackage{mathtools}
\usepackage{amsthm}
\usepackage{amsmath}
\usepackage{amssymb}
\usepackage{amstext}
\usepackage{latexsym}
\usepackage{dsfont}
\usepackage[framemethod=TikZ]{mdframed}

\usepackage{makeidx}
\makeindex

\usepackage{custom}



\begin{document}

\begin{frame}
  \titlepage
\end{frame}

\begin{frame}{Tabla de Contenidos}
  \tableofcontents
\end{frame}

\section{Introducción}

    \begin{frame}
        \frametitle{Tabla de Contenidos}
        \tableofcontents[currentsection]
    \end{frame}

    \subfile{01-introduccion/01-introduccion.tex}
    
\section{Definiciones}
    \begin{frame}
        \frametitle{Tabla de Contenidos}
        \tableofcontents[currentsection]
    \end{frame}

    \subfile{02-definiciones/02-definiciones.tex}
    
\section{Sección 3}
    \begin{frame}
        \frametitle{Tabla de Contenidos}
        \tableofcontents[currentsection]
    \end{frame}

    \subfile{03-seccion03/03-seccion03.tex}
    
\section{Sección 4}
    \begin{frame}
        \frametitle{Tabla de Contenidos}
        \tableofcontents[currentsection]
    \end{frame}

    \subfile{04-seccion04/04-seccion04.tex}
    
\section{Sección 5}
    \begin{frame}
        \frametitle{Tabla de Contenidos}
        \tableofcontents[currentsection]
    \end{frame}
    
    \subfile{05-seccion05/05-seccion05.tex}

\end{document}
